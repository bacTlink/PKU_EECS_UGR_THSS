% Copyright (c) 2019 Bochen Tan
% Public domain.
%本模板的宗旨是尽量绿色,不需要附加安装任何东西。
%按照教务部下发的WORD说明文档格式,下简称“说明”
%没有封面和评阅表,这两部分请直接在Cover&ReviewTable.doc中写再输出pdf拼到一起
%doc小改动:封面校徽和文字替换为了高清版本,“题目:”和中文题目对齐,中英文题目分在了表的两行
%doc小改动:插入了两个白页,使得连续打印的时候封面和表格都在奇数页
%正文部分改动:在每一页下方中央加了页码,因为说明中页眉不分奇偶页,所以页码就都在中央吧
%不含自动的参考文献,因为说明中参考文献格式不典型,请手动输入或自行写程序
%在Windows或Linux下渲染出字体更接近说明,Mac OS上字体不太一样
%有警告\headheight is too small,fancyhdr的上距离有点小,似乎问题不大

\documentclass[UTF8,openany,AutoFakeBold,AutoFakeSlant,cs4size]{ctexbook}
%openany 使一章可以从偶数页开始,因为说明中每一章并没有只能从奇数页开始,虽然这是常理
%AutoFakeBold 和 AutoFakeSlant 因为 CJK 里没有真正的加粗和倾斜,如果额外字体则效果更好
%cs4size 因为要求主题是小四号字

\usepackage[a4paper,left=2.6cm,right=2.6cm,top=3cm,bottom=2.5cm]{geometry}
%自定义的页边距

\usepackage{fontspec}

\usepackage{amsmath}
\usepackage{bm}
\usepackage{amsfonts}
\usepackage{enumerate}
\usepackage{fancyhdr}

\usepackage{changepage}   % for the adjustwidth environment

\usepackage{cite}
\newcommand{\upcite}[1]{\textsuperscript{\cite{#1}}} %引用在右上角

\usepackage{multirow,booktabs,makecell}
\usepackage{graphicx}
\usepackage[font=small,labelsep=space]{caption} %五号,宋体/Time new roman
\renewcommand{\thetable}{\arabic{table}} %表格和图片编号不分章节,直接1,2,3 ...
\renewcommand{\thefigure}{\arabic{figure}}
\renewcommand{\theequation}{\arabic{chapter}.\arabic{equation}} %公式标签 章.公式(均为阿拉伯数字)

\usepackage{chngcntr}
\counterwithin{figure}{chapter} %X.X的图像编号
\counterwithin{table}{chapter} %X.X的表格编号

\usepackage{tocloft} %自定义目录,说明中没有明确规定,和WORD自动生成目录格式一致

%“目录”两个字的格式
\renewcommand\cftbeforetoctitleskip{0pt}
\renewcommand\cftaftertoctitleskip{0pt}
\renewcommand\cfttoctitlefont{\heiti\zihao{3}}

\renewcommand\cftchapfont{\heiti\normalsize} %黑体小四
\renewcommand\cftchapdotsep{\cftdotsep} %有点连到页码,点间距不确定,待改
\renewcommand\cftchappagefont{\heiti\normalsize} %黑体小四页码
\renewcommand\cftbeforechapskip{0pt}

%1. 第一级 五号宋体,缩进两个字符,页码一致
\renewcommand\cftsecfont{\songti\small}
\renewcommand\cftsecpagefont{\songti\small}
\renewcommand\cftsecaftersnum{} %一级目录号后加点
\renewcommand\cftsecindent{2em}
\renewcommand\cftbeforesecskip{0pt}

%1.1 第二级 五号宋体,缩进四个字符,页码一致
\renewcommand\cftsubsecfont{\songti\small}
\renewcommand\cftsubsecpagefont{\songti\small}
\renewcommand\cftsubsecindent{4em}
\renewcommand\cftbeforesubsecskip{0pt}

%1.1.1 第二级 五号宋体,缩进四个字符,页码一致
\renewcommand\cftsubsubsecfont{\songti\small}
\renewcommand\cftsubsubsecpagefont{\songti\small}
\renewcommand\cftsubsubsecindent{4em}
\renewcommand\cftbeforesubsubsecskip{0pt}

\setsansfont{Arial}

\usepackage{titlesec}%自定义章节标题
\CTEXsetup[format={\center\heiti\zihao{3}\sffamily},beforeskip=0pt]{chapter}

%第一章  绪论(二号、黑体) beforeskip为上方垂直距离看起来还比说明偏大,待改

\setcounter{tocdepth}{3}
\setcounter{secnumdepth}{3}
%使目录中有三级标题,即subsubsection

%\renewcommand\thesection{\arabic{section}} % 使得不显示章名,只显示节名
\titleformat{\section}
{\raggedright\zihao{3}\bfseries\heiti}
{\thesection\quad}
{0pt}
{}%1. 第一级(三号、宋体/Time new roman、加粗)

\titleformat{\subsection}
{\raggedright\bfseries\zihao{4}\songti}
{\thesubsection\quad}
{0pt}
{}%1.1 第二级(四号,宋体/Time new roman,加粗)

\titleformat{\subsubsection}
{\raggedright\bfseries\zihao{-4}\songti}
{\thesubsubsection\quad}
{0pt}
{}%1.1.1 第三级(小四,宋体/Time new roman,加粗)


% 封面依赖的宏包
\usepackage{xcoffins} % 用于设计封面格式
\usepackage{xcolor}
\usepackage{xeCJK} % 用于引入楷体
\usepackage{soul} % 用于设置下划线宽度
\setul{}{2pt}
\setmainfont{Times New Roman} % Times New Roman 作为默认英文字体
% 引入楷体,请改成自己系统里对应的名字
\setCJKfamilyfont{kaiti}[AutoFakeBold=1.5]{AR PL KaitiM GB}
\newcommand{\kaiti}{\CJKfamily{kaiti}}


\title{}
\author{}
\date{}
\begin{document}

% 封面中需要修改的内容直接在此处更改即可
\newcommand{\chineseTitle}{题目第一行}
\newcommand{\chineseTitleSecondLine}{题目第二行}
\newcommand{\englishTitle}{Efficient Global Illumination Method based on}
\newcommand{\englishTitleSecondLine}{Statistical Hypothesis Tests over Samples}
\newcommand{\name}{姓名}
\newcommand{\studentID}{学号}
\newcommand{\school}{学院}
\newcommand{\researchDirection}{专业}
\newcommand{\major}{方向}
\newcommand{\advisor}{XX教授}
% 插入封面
% 声明需要的Coffin
\NewCoffin \result
\NewCoffin \topBox
\NewCoffin \badge
\NewCoffin \pku
\NewCoffin \headingText
\NewCoffin \titleText
\NewCoffin \chineseTitleText
\NewCoffin \chineseTitleTextSecondLine
\NewCoffin \englishTitleText
\NewCoffin \englishTitleTextSecondLine
\NewCoffin \nameText
\NewCoffin \studentIDText
\NewCoffin \schoolText
\NewCoffin \majorText
\NewCoffin \researchDirectionText
\NewCoffin \advisorText
\NewCoffin \dateText


% 各个Coffin的内容
\SetHorizontalCoffin \result {}
\SetHorizontalCoffin \topBox {\color{white} \rule{210mm}{41mm}}
\SetHorizontalCoffin \badge {\includegraphics[width=20.6mm]{badge}}
\SetHorizontalCoffin \pku {\includegraphics[width=60.5mm]{pku}}
\SetVerticalCoffin \headingText{160mm}{\center\heiti\fontsize{36}{36}\textcolor{black}{硕士研究生毕业论文}}
\SetVerticalCoffin \titleText{25.4mm}{\songti\zihao{2}{题目:}}
\SetVerticalCoffin \chineseTitleText{110mm}{\bfseries\heiti\zihao{1}{\underline{\makebox[112mm][l]{\chineseTitle}}}}
\SetVerticalCoffin \chineseTitleTextSecondLine{110mm}{\bfseries\heiti\zihao{1}{\underline{\makebox[112mm][l]{\chineseTitleSecondLine}}}}
\SetVerticalCoffin \englishTitleText{110mm}{\bfseries\kaiti\zihao{3}\underline{\makebox[112mm][l]{\englishTitle}}}
\SetVerticalCoffin \englishTitleTextSecondLine{110mm}{\bfseries\kaiti\zihao{3}\underline{\makebox[112mm][l]{\englishTitleSecondLine}}}
\SetVerticalCoffin \nameText{104mm}{\center\heiti\fontsize{15}{15}\textcolor{black}{姓\qquad\ \ \ 名:\underline{\makebox[76mm][c]{\fangsong{\name}}}}}
\SetVerticalCoffin \studentIDText{104mm}{\center\heiti\fontsize{15}{15}\textcolor{black}{学\qquad\ \ \ 号:\underline{\makebox[76mm][c]{\fangsong\zihao{3}\studentID}}}}
\SetVerticalCoffin \schoolText{104mm}{\center\heiti\fontsize{15}{15}\textcolor{black}{院\qquad\ \ \ 系:\underline{\makebox[76mm][c]{\fangsong{\school}}}}}
\SetVerticalCoffin \majorText{104mm}{\center\heiti\fontsize{15}{15}\textcolor{black}{专\qquad\ \ \ 业:\underline{\makebox[76mm][c]{\fangsong{\major}}}}}
\SetVerticalCoffin \researchDirectionText{104mm}{\center\heiti\fontsize{15}{15}\textcolor{black}{研究方向:\underline{\makebox[76mm][c]{\fangsong{\researchDirection}}}}}
\SetVerticalCoffin \advisorText{104mm}{\center\heiti\fontsize{15}{15}\textcolor{black}{导师姓名:\underline{\makebox[76mm][c]{\fangsong{\advisor}}}}}
\SetVerticalCoffin \dateText{104mm}{\center{\songti\fontsize{15}{15}\textcolor{black} 二〇二一\quad 年\quad 五\quad 月}}


% 指定各个Coffin相对位置关系
\JoinCoffins \result \topBox
\JoinCoffins \result[\topBox-hc, \topBox-b] \badge[r, b](-27.9mm, -20.6mm)
\JoinCoffins \result[\topBox-hc, \topBox-b] \pku[l, b](-11.3mm, -20.6mm)
\JoinCoffins \result[\topBox-hc, \topBox-b] \headingText[hc, b](0mm, -63.8mm)
\JoinCoffins \result[\headingText-hc, \headingText-b] \titleText[l, t](-83.25mm, -20mm)
\JoinCoffins \result[\headingText-hc, \headingText-b] \chineseTitleText[l, t](-60.85mm, -18.85mm)
\JoinCoffins \result[\headingText-hc, \headingText-b] \chineseTitleTextSecondLine[l, t](-60.85mm, -30.85mm)
%\JoinCoffins \result[\headingText-hc, \headingText-b] \englishTitleText[l, t](-49.85mm, -40mm)
%\JoinCoffins \result[\headingText-hc, \headingText-b] \englishTitleTextSecondLine[l, t](-49.85mm, -48mm)
\JoinCoffins \result[\headingText-hc, \headingText-b] \nameText[hc, t](0mm, -60mm)
\JoinCoffins \result[\nameText-hc, \nameText-b] \studentIDText[hc, t](0mm, 0mm)
\JoinCoffins \result[\studentIDText-hc, \studentIDText-b] \schoolText[hc, t](0mm, 0mm)
\JoinCoffins \result[\schoolText-hc, \schoolText-b] \majorText[hc, t](0mm, 0mm)
\JoinCoffins \result[\schoolText-hc, \schoolText-b] \researchDirectionText[hc, t](0mm, -10mm)
\JoinCoffins \result[\majorText-hc, \majorText-b] \advisorText[hc, t](0mm, -10mm)
\JoinCoffins \result[\advisorText-hc, \advisorText-b] \dateText[hc, t](0mm, -20mm)


% 输出封面
\thispagestyle{empty}
\newgeometry{left=0mm,bottom=0mm, top=0mm, right=0mm}
\noindent\TypesetCoffin \result
\restoregeometry
\clearpage

% 插入导师评阅表
% \input{ReviewTable}
% \clearpage


\linespread{1.5}\selectfont
\chapter*{版权声明}
% 本页不计页码
\thispagestyle{empty}
% 本页无页眉和页脚
任何收存和保管本论文各种版本的单位和个人,未经本论文作者同意,不得将本论文转借他人,亦不得随意复制、抄录、拍照或以任何方式传播。否则,引起有碍作者著作权之问题,将可能承担法律责任。
\clearpage

%版权声明后空白一页,使得摘要从奇数页开始。
% \quad
% \setcounter{page}{0}
% 本页不计页码
% \thispagestyle{empty}
% 本页无页眉和页脚
% \clearpage



\pagestyle{fancy}
\normalsize
\linespread{1.5}\selectfont
%小四号,宋体/Time new roman,1.5倍行距
\chapter*{摘要}
\pagenumbering{Roman}
\setcounter{page}{1}

摘要

\vspace*{\fill}
\noindent{\songti {关键词}: A,B,C}



%\addcontentsline{toc}{chapter}{摘要} %手动加入目录
\fancypagestyle{plain} %因为latex默认每章第一页是plain所以需要重置一下plain和说明统一
{
	\fancyhf{} %清空

	\fancyhead[CO]{摘要}
	%偶数页右页眉,奇数页右页眉均为“摘要”,及章名\leftmark

	\fancyhead[CE]{北京大学硕士学位论文}
	%偶数页左页眉,奇数页左页眉均为“北京大学硕士学位论文”

	\fancyfoot[CO,CE]{~\thepage~}
	%偶数页和奇数页中页脚为页码,从对称考虑,因为每页在说明中都是一样的,不分奇偶

	\renewcommand{\headrulewidth}{0.7pt} %页眉线宽度,可调,不太清楚说明中是多少,待改

	\renewcommand{\footrulewidth}{0pt} %页脚线宽度为0,即没有
}

%默认的风格是fancy,设置于下,用于每章非第一页
\fancyhf{}
\fancyhead[CO]{摘要}
\fancyhead[CE]{北京大学硕士学位论文}
\fancyfoot[CO,CE]{~\thepage~}
\renewcommand{\headrulewidth}{0.7pt}
\renewcommand{\footrulewidth}{0pt}
\clearpage





%5号,Time new roman,1.5倍行距
\chapter*{Title}

\vspace{-0.3in}
\begin{center}
\normalsize
Name (Major)
\\
Directed by Director
\\
\vspace{0.15in}
{\sffamily \bfseries
ABSTRACT}
\end{center}

Abstract

\vspace*{\fill}
\noindent{
{KEY WORDS}: Abc, Bcd, Etc}



%\addcontentsline{toc}{chapter}{\bfseries Abstract} %Abstract加粗
\fancypagestyle{plain}
{
	\fancyhf{}
	\fancyhead[CO]{Abstract}
	\fancyhead[CE]{北京大学硕士学位论文}
	\fancyfoot[CO,CE]{~\thepage~}
	\renewcommand{\headrulewidth}{0.7pt}
	\renewcommand{\footrulewidth}{0pt}
}
\fancyhf{}
\fancyhead[CO]{Abstract}
\fancyhead[CE]{北京大学硕士学位论文}
\fancyfoot[CO,CE]{~\thepage~}
\renewcommand{\headrulewidth}{0.7pt}
\renewcommand{\footrulewidth}{0pt}
\clearpage





\fancypagestyle{plain}
{
	\fancyhf{}
	\fancyhead[CO]{目录}
	\fancyhead[CE]{北京大学硕士学位论文}
	\fancyfoot[CO,CE]{~\thepage~}
	\renewcommand{\headrulewidth}{0.7pt}
	\renewcommand{\footrulewidth}{0pt}
}
\fancyhf{}
\fancyhead[CO]{目录}
\fancyhead[CE]{北京大学硕士学位论文}
\fancyfoot[CO,CE]{~\thepage~}
\renewcommand{\headrulewidth}{0.7pt}
\renewcommand{\footrulewidth}{0pt}
\renewcommand{\contentsname}{\centerline{\heiti\zihao{3}{目录}}\vspace{0.3in}}
\tableofcontents
%\addcontentsline{toc}{chapter}{目录}
\clearpage



\normalsize
\linespread{1.5}\selectfont
%正文,小四号,中文宋体,英文Time new roman,1.5倍行距
\fancypagestyle{plain}
{
	\fancyhf{}
	\fancyhead[CO]{\leftmark}
	\fancyhead[CE]{北京大学硕士学位论文}
	\fancyfoot[CO,CE]{~\thepage~}
	\renewcommand{\headrulewidth}{0.7pt}
	\renewcommand{\footrulewidth}{0pt}
}
\fancyhf{}
\fancyhead[CO]{\leftmark}
\fancyhead[CE]{北京大学硕士学位论文}
\fancyfoot[CO,CE]{~\thepage~}
\renewcommand{\headrulewidth}{0.7pt}
\renewcommand{\footrulewidth}{0pt}

\chapter*{主要符号对照表}
可有可无
%\addcontentsline{toc}{chapter}{主要符号对照表}
\clearpage

\setcounter{page}{1}
\pagenumbering{arabic}

\chapter{引言}
引言

\chapter{章节名称}
\section{一级段落名称}
\subsection{二级段落名称}
引用如\upcite{lin2020cppm},引用表如表\ref{tab:input_output_r},引用图如图\ref{fig:sample}.


\section{一级段落名称2}
\subsection{二级段落名称2}

\begin{table}[h]
\small %内容,(五号,宋体/Time new roman)
\centering
\caption{不同频率下的输入和输出阻抗}
\label{tab:input_output_r}
\begin{tabular}{cccc} %表格使用三线表
\toprule %不确定说明中三条线的粗细,待改
频率(Hz) & 1 & 10k & 1M \\
\midrule
输入电阻($\Omega/^\circ$) & 339.719k/-87.84 & 5.6707k/-9.827 & 351.188/-72.377\\
输出电阻($\Omega/^\circ$) & 338.638k/-89.663 & 1.9866k/-1.1228 & 1.9189k/-14.801 \\
\bottomrule
\end{tabular}
\end{table}

\begin{figure}[h]
\centering
\includegraphics[width=12cm]{Sample.jpg}
\caption{示例图片}
\label{fig:sample}
\end{figure}
\clearpage

\small
\linespread{1}\selectfont
%正文,五号,中文宋体,英文Time new roman,1倍行距
\chapter*{参考文献}
请参考pdf要求

\addcontentsline{toc}{chapter}{参考文献}
\fancypagestyle{plain}
{
	\fancyhf{}
	\fancyhead[CO]{参考文献}
	\fancyhead[CE]{北京大学硕士学位论文}
	\fancyfoot[CO,CE]{~\thepage~}
	\renewcommand{\headrulewidth}{0.7pt}
	\renewcommand{\footrulewidth}{0pt}
}
\fancyhf{}
\fancyhead[CO]{参考文献}
\fancyhead[CE]{北京大学硕士学位论文}
\fancyfoot[CO,CE]{~\thepage~}
\renewcommand{\headrulewidth}{0.7pt}
\renewcommand{\footrulewidth}{0pt}






\bibliographystyle{myunsrt}
\bibliography{ref}
这是参考“参考文献”,主要用来看引用的顺序,请手动些参考文献或自行写程序,最终编译请删除
\clearpage




\linespread{1.5}\selectfont
\normalsize
%正文,小四号,中文宋体,英文Time new roman,1.5倍行距
\chapter*{致谢}



\addcontentsline{toc}{chapter}{致谢}
\fancypagestyle{plain}
{
	\fancyhf{}
	\fancyhead[CO]{致谢}
	\fancyhead[CE]{北京大学硕士学位论文}
	\fancyfoot[CO,CE]{~\thepage~}
	\renewcommand{\headrulewidth}{0.7pt}
	\renewcommand{\footrulewidth}{0pt}
}
\fancyhf{}
\fancyhead[CO]{致谢}
\fancyhead[CE]{北京大学硕士学位论文}
\fancyfoot[CO,CE]{~\thepage~}
\renewcommand{\headrulewidth}{0.7pt}
\renewcommand{\footrulewidth}{0pt}




{\linespread{1}\selectfont
\normalsize
%小四号,中文宋体,英文Time new roman,1倍行距
\chapter*{北京大学学位论文原创性声明和使用授权说明}}

\begin{center}
\songti \bfseries \zihao{4}
原创性声明
\end{center}

{
\songti 
本人郑重声明:所呈交的学位论文,是本人在导师的指导下,独立进行研究工作所取得的成果。除文中已经注明引用的内容外,本论文不含任何其他个人或集体已经发表或撰写过的作品或成果。对本文的研究做出重要贡献的个人和集体,均已在文中以明确方式标明。本声明的法律结果由本人承担。

\quad

\hspace*{0.22\linewidth} 论文作者签名:\quad\quad\quad\quad\quad 日期:\quad\quad\; 年\quad \; 月\quad \; 日
}

\quad

\begin{center}
\songti \bfseries \zihao{4}
学位论文使用授权说明
\end{center}
\vspace*{-0.4in}
\begin{center}
\songti \fontsize{9}{9}
(必须装订在提交学校图书馆的印刷本)
\end{center}

{
\songti
{
本人完全了解北京大学关于收集、保存、使用学位论文的规定,即:
\renewcommand{\labelitemi}{●}
\begin{itemize}
    \setlength\itemsep{0.0in}
    \item \;\;按照学校要求提交学位论文的印刷本和电子版本;
    \item \;\;学校有权保存学位论文的印刷本和电子版,并提供目录检索与阅览服务,在校园网上提供服务;
    \item \;\;学校可以采用影印、缩印、数字化或其它复制手段保存论文;
    \item \;\;因某种特殊原因需要延迟发布学位论文电子版,授权学校{\raisebox{-0.02in}{\hspace{0.01in}\zihao{2}□\hspace{0.01in}}}一年/{\raisebox{-0.02in}{\hspace{0.01in}\zihao{2}□\hspace{0.01in}}}两年/{\raisebox{-0.02in}{\hspace{0.01in}\zihao{2}□\hspace{0.01in}}}三年以后,在校园网上全文发布。
\end{itemize}
}

}

\quad

{\songti
\hspace*{0.2\linewidth} (保密论文在解密后遵守此规定)
}

\quad


\quad


\quad


{
\songti
\hspace*{0.22\linewidth} 论文作者签名:\quad\quad\quad\quad\quad 导师签名:

\hspace*{0.3\linewidth} 日期:\quad\quad\; 年\quad \; 月\quad \; 日
}

\addcontentsline{toc}{chapter}{北京大学学位论文原创性声明和使用授权说明}
\fancypagestyle{plain}
{
	\fancyhf{}
	\fancyhead[CO]{北京大学学位论文原创性声明和使用授权说明}
	\fancyhead[CE]{北京大学硕士学位论文}
	\fancyfoot[CO,CE]{~\thepage~}
	\renewcommand{\headrulewidth}{0.7pt}
	\renewcommand{\footrulewidth}{0pt}
}
\fancyhf{}
\fancyhead[CO]{北京大学学位论文原创性声明和使用授权说明}
\fancyhead[CE]{北京大学硕士学位论文}
\fancyfoot[CO,CE]{~\thepage~}
\renewcommand{\headrulewidth}{0.7pt}
\renewcommand{\footrulewidth}{0pt}
\clearpage


\end{document}
